\documentclass[10pt]{article}

\usepackage[utf8]{inputenc} 
\usepackage[russian]{babel} 
\usepackage[usenames]{color}
\usepackage{amsmath, amssymb, colortbl, hyperref, fancyhdr}
\usepackage[dvipdfmx]{graphicx}
\usepackage[a4paper, margin=.5in, top=1in]{geometry}

\pagestyle{fancy}

\newcommand{\N}{\mathbb{N}}
\newcommand{\Z}{\mathbb{Z}}
\newcommand{\Q}{\mathbb{Q}}
\newcommand{\R}{\mathbb{R}}
\renewcommand{\C}{\mathbb{C}}

\renewcommand{\le}{\leqslant}
\renewcommand{\leq}{\leqslant}
\renewcommand{\ge}{\geqslant}
\renewcommand{\geq}{\geqslant}

\renewcommand{\O}{\varnothing}
\def\xor{\oplus}
\def\ol{\overline{}}
\renewcommand{\headrulewidth}{1pt}
\lhead{Фидбеки}\chead{}\rhead{Page \thepage\ of \pageref{LastPage}}
\lfoot{}\cfoot{}\rfoot{}

\begin{document}

\section{Аля}

Биномиальный коэффициент:

\begin{verbatim}
WithFun
Fun Fact With n ThatsAll
  Seq
    Assign ans 1
    While > n 0
      Seq
        Assign ans * ans n
        Assign n - n 1
  Return ans

ThatsAllFun

Seq
  Read n
  Seq
    Read k
    Write / / Call Fact With n ThatsAll Call Fact With k ThatsAll Call Fact With - n k
  \end{verbatim}

Фибоначчи:

\begin{verbatim}
WithFun
Fun Fib With n ThatsAll
  Seq
    Assign f 0
    Seq
      Assign s 1
      While > n 0
        Seq
          Assign t + f s
          Seq
            Assign f s
            Seq
              Assign s t
              Assign n - n 1
  Return f

ThatsAllFun

Seq
  Read n
  Write Call Fib With n ThatsAll
\end{verbatim}



Удвоение числа:

\begin{verbatim}
WithFun
Fun Sum With a With b ThatsAll
  While > b 0
    Seq
      Assign a + a 1
      Assign b - b 1
  Return a

WithFun
Fun Minus With a With b ThatsAll
  While > b 0
    Seq
      Assign a - a 1
      Assign b - b 1
  Return a

ThatsAllFun

Seq
  Read n
  Write Call Sum With Call Sum With n With 5 ThatsAll With Call Minus With n With 5 ThatsAll ThatsAll
\end{verbatim}


Функции были добавлены очень органично. Разобраться с ними было сильно легче, чем изначально с языком.



\section{Олень}

Биномиальный коэффициент:

\begin{verbatim}
fun fact: n
./
    bind ans (1);
    while (n>0)
    ./
        bind ans (ans*n);
        bind n (n-1);
    \.;
\.
up (ans)

./
    read n;
    read k;
    write (fact(n)/(fact(k)*fact(n-k)));
\.
\end{verbatim}

Фибоначчи:

\begin{verbatim}
fun fib: n
./
    bind f (0);
    bind s (1);
    while (n>0)
    ./
        bind t (f+s);
        bind f (s);
        bind s (t);
        bind n (n-1);
    \.;
\.
up (f)


./
    read n;
    write (fib(n));
\.
\end{verbatim}

Удвоение числа:

\begin{verbatim}
fun sum: a b
./
    while (b>0)
    ./
        bind a (a+1);
        bind b (b-1);
    \.;
\.
up (a)

fun minus: a b
./
    while (b>0)
    ./
        bind a (a-1);
        bind b (b-1);
    \.;
\.
up (a)

./
    read n;
    write (sum(sum(n, 5), minus(n, 5)));
\.
\end{verbatim}



Очень удобно все так, параметры просто по очереди без запятых~--- это прикольно.

\label{LastPage}
\end{document}


