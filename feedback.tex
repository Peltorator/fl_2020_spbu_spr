\documentclass[10pt]{article}

\usepackage[utf8]{inputenc} 
\usepackage[russian]{babel} 
\usepackage[usenames]{color}
\usepackage{amsmath, amssymb, colortbl, hyperref, fancyhdr}
\usepackage[dvipdfmx]{graphicx}
\usepackage[a4paper, margin=.5in, top=1in]{geometry}

\pagestyle{fancy}

\newcommand{\N}{\mathbb{N}}
\newcommand{\Z}{\mathbb{Z}}
\newcommand{\Q}{\mathbb{Q}}
\newcommand{\R}{\mathbb{R}}
\renewcommand{\C}{\mathbb{C}}

\renewcommand{\le}{\leqslant}
\renewcommand{\leq}{\leqslant}
\renewcommand{\ge}{\geqslant}
\renewcommand{\geq}{\geqslant}

\renewcommand{\O}{\varnothing}
\def\xor{\oplus}
\def\ol{\overline{}}
\renewcommand{\headrulewidth}{1pt}
\lhead{Фидбеки}\chead{}\rhead{Page \thepage\ of \pageref{LastPage}}
\lfoot{}\cfoot{}\rfoot{}

\begin{document}

\section{Аля}

Факториал числа:

\begin{verbatim}
Seq
  Read n
  Seq
    Assign ans 1
    Seq
      While > n 0
        Seq
          Assign ans * ans n
          Assign n - n 1
      Write ans
\end{verbatim}

Фибоначчи:

\begin{verbatim}
Seq
  Read n
  Seq
    Assign f 0
    Seq
      Assign s 1
      Seq
        While > n 0
          Seq
            Assign t + f s
            Seq
              Assign f s
              Seq
                Assign s t
                Assign n - n 1
        Write f
\end{verbatim}


Нашел маленький баг в документации. Там было написано, что Read нужно пользоваться в виде

\begin{verbatim}
    V0 Read
\end{verbatim}

Хотя на самом деле надо пользоваться так:

\begin{verbatim}
    Read V0
\end{verbatim}

Язык очень удобный, потому что не надо ставить всякие вспомогательные ключевые слова типа фигурных скобок.
Но следствием из этого является проблема с Seq. Неклассно его везде писать. И Nop~--- весьма странный костыль.
Могут быть проблемы с написанием больших арифметических выражениях из-за польской записи.
В остальном\ldots Если ставить отступы, то очень удобно писать на языке.





\section{Олень}

Факториал числа:

\begin{verbatim}
./
    read n;
    bind ans (1);
    while (n>0)
    ./
        bind ans (ans*n);
        bind n (n-1);
    \.;
    write (ans);
\.
\end{verbatim}

Фибоначчи:

\begin{verbatim}
./
    read n;
    bind f (0);
    bind s (1);
    while (n>0)
    ./
        bind t (ans+ans');
        bind f (s);
        bind s (t);
        bind n (n-1);
    \.;
    write (f);
\.
\end{verbatim}



Из-за ограничения на длину назания переменных наверняка будут проблемы с тем,
что названия переменных~--- непонятный случайный набор букв.
Использование странных наборов символов вместо фигурных скобок выглядит сомнительным решением,
потому что надо писать два символа, но окей\ldots
В выражениях нельзя ставить пробелы :( Это очень грустно.
В остальном все замечательно. Писать точно удобнее, чем на предыдущем языке, потому что не надо бесконечно думать о Seq'ах.


\label{LastPage}
\end{document}


