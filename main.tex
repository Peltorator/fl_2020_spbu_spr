\documentclass[12pt]{article}
\usepackage[left=2cm,right=2cm,top=2cm,bottom=2cm,bindingoffset=0cm]{geometry}
\usepackage[utf8x]{inputenc}
\usepackage[english,russian]{babel}
\usepackage{cmap}
\usepackage{amssymb}
\usepackage{amsmath}
\usepackage{url}
\usepackage{pifont}
\usepackage{tikz}
\usepackage{verbatim}

\usetikzlibrary{shapes,arrows}
\usetikzlibrary{positioning,automata}
\tikzset{every state/.style={minimum size=0.2cm},
initial text={}
}


\newenvironment{myauto}[1][3]
{
  \begin{center}
    \begin{tikzpicture}[> = stealth,node distance=#1cm, on grid, very thick]
}
{
    \end{tikzpicture}
  \end{center}
}



\begin{document}

\begin{section}{Язычок}
    См. peLLang.md
\end{section}

\begin{section}{Нормальная форма Хохомского}
    / Жирный шрифт отображает изменения, но $\varepsilon$ и тд сделать жирными у меня не получилось к сожалению /

\

    Было:

    S $\to$ RS | R
    
    R $\to$ aSb | cRd | ab | cd | $\varepsilon$

    \

    Проведем нормализацию.
    Уберем правила, в которых есть и терминалы, и нетерминалы:

    S $\to$ RS | R
    
    R $\to$ \textbf{A}S\textbf{B} | \textbf{C}R\textbf{D} | \textbf{AB} | \textbf{CD} | $\varepsilon$

    \textbf{A $\to$ a}

    \textbf{B $\to$ b}

    \textbf{C $\to$ c}

    \textbf{D $\to$ d}


\

    Теперь надо выгнать длинные правила, разбив их на кусочки:

    S $\to$ RS | R
    
    R $\to$ \textbf{X}B | \textbf{Y}D | AB | CD | $\varepsilon$

    \textbf{X $\to$ AS}

    \textbf{Y $\to$ CR}

    A $\to$ a

    B $\to$ b

    C $\to$ c

    D $\to$ d

\

    S $\to$ R $\to \varepsilon$, так что надо добавить правило \textbf{S} $\to \varepsilon$, а также распространить это на X и Y:

    X $\to$ AS \textbf{| A}
    
    Y $\to$ CR \textbf{| C}

\

    Теперь можно можно удалить $\varepsilon$-правила, но также можно заметить, что правило X $\to$ A не нужно, потому что S
    все еще можно превратить в $\varepsilon$:


    S $\to$ RS | R | $\varepsilon$
    
    R $\to$ XB | YD | AB | CD

    X $\to$ AS

    Y $\to$ CR | C

    A $\to$ a

    B $\to$ b

    C $\to$ c

    D $\to$ d


\

    Нам нельзя иметь стартовое в правой части, так что делаем новое стартовое S':

    \textbf{S' $\to$ S | $\varepsilon$}

    S $\to$ RS | R
    
    R $\to$ XB | YD | AB | CD

    X $\to$ AS

    Y $\to$ CR | C

    A $\to$ a

    B $\to$ b

    C $\to$ c

    D $\to$ d

\

    От правил вида M $\to$ N нужно тоже избавиться по принципу распространения правил:


    S' $\to$   \textbf{RS | XB | YD | AB | CD} | $\varepsilon$

    S $\to$ RS | \textbf{XB | YD | AB | CD}
    
    R $\to$ XB | YD | AB | CD

    X $\to$ AS

    Y $\to$ CR | \textbf{c}

    A $\to$ a

    B $\to$ b

    C $\to$ c

    D $\to$ d


\

    Это и будет финальным результатом.

\end{section}


\begin{section}{CS:GO язык}
    
    Чтобы доказать, что он КС, построим автомат:

\begin{myauto}
    \node[state,initial]   (s)              {$S$};
    \node[state]           (p) [below=of s] {$P$};
    \node[state,accepting] (q) [left=of  p] {$Q$};
    \node[state]           (x) [right=of s] {$X$};
    \node[state,accepting] (y) [below=of x] {$Y$};
    \node[state,accepting] (t) [right=of x] {$T$};
    \node[state]           (z) [right=of y] {$Z$};

    \path[->] (s) edge node [left]  {$b$} (p)
              (p) edge node [above] {$b$} (q)
              (q) edge node [above] {$b$} (p)
              (s) edge node [above] {$a$} (x)
              (x) edge node [above] {$b$} (t)
              (x) edge node [left]  {$a$} (y)
              (y) edge node [left]  {$a$} (x)
              (y) edge node [above] {$b$} (z)
              (z) edge node [left]  {$b$} (t)
              (t) edge node [left]  {$b$} (z)
    ;
    \end{myauto}
    
    Q отвечает за строки вида $b^{2k}$, где $k > 0$. Y отвечает за строки вида $a^{2k}$, где $k > 0$.
    T отвечает за строки, в которых есть и $a$, и $b$.

\end{section}

\label{LastPage}
\end{document}

